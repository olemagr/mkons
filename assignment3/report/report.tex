
\documentclass[11pt]{report}
\usepackage{graphicx}
\usepackage{url}
\usepackage{amsmath}
\title{TDT4255 Assignment 3}
\author{Knut Halvor Skrede \& Ole Magnus Ruud}
\begin{document}
\maketitle
\clearpage


\section*{Abstract/Summary}
        
% Lab-report presents documentation of your work which was done during the exercise. 
% It is important
% that the reader can understand what the group has done and how it came to the solution.
% In short, abstract should contain overview of:
% – what the task of the exercise was
% – how it was solved
% – what works
% – what does not work
% – if something extra was done by the group
% – if something should have been done in another way

The task was to implement a simple multicycle CPU. A top level architecture 
was proposed in the assignment text, we chose to implement the CPU in this way. 
We could use the ALU we made from the previous assignment, but we chose to use
the one included in the support files. The CPU should load instructions into 
memory and execute them. A program counter will be used to walk through the 
instructions. A BNZ (branch if not zero) instruction can also be used, this 
instruction will load the program counter with a given address if the zero flag
in the status register is not set. Additionally the CPU should implement a 
LDI (load immediate) instruction that loads the the register with a value. 
The cpu should also implement the ALU functionality.

We chose to use the instruction word layout proposed in the assignment text.

The control unit implements a simple fetch-execute state machine.

The assignment was also to write a simple program to run on the CPU, 
we chose to implement fibonacci.

\section*{Introduction}

% Present what the task within the exercise is and what challenges it
% gives.  For example, which kind of program has to be written and
% which hardware has to be used.  Short introduction to how the group
% has solved this task.  It is important that what has been
% accomplished is clearly presented.  If there is something in the
% exercise what has not been completed or what does not function as it
% should, it should also be described.  If applicable, write on the
% motivation and solution(s) for extra functionality which the group
% has set in.

The task was essentially to create a CPU. We had to implement the PC
register, the status register, muxers for PC register and regfile and
the unit controlling the write enable signals of the registers and
select signals for the muxers.  The main register and memory was given
to us.

\section*{Suggested Solution}

% Present concisely an outline of how the task was solved.  This is
% the place to write in greater detail about how the group was
% developing ideas for achieving solution.  Add an outline of, for
% example, how hardware is built up and how it functions.  It is
% advised to use concise and clear flow diagrams, UML or block
% diagram.  All code should be delivered in a separate file. It is
% important that documentation contains enough description of the
% program so that the reader can understand how the program works and
% what files with the source code are attached. Write about how the
% group has come to the solution, what tests were used to ensure that
% the program runs as it should, for example, what documentation the
% group has used, possibly some additional sources which were helpful
% to do the exercise.  Remember that documentation is a part of the
% grade. So, it is not enough to have a brilliant solution if the
% reader can't understand how it is meant to work.

\subsection*{Approach}

We started by making skeleton modules for all the components, mapping
the signals between them. After that we implemented the simplest
components. To make sure nothing was wrong before writing the more
difficult components, we wrote simple testbenches for them. We then
proceeded to implement the control unit, which we considered to be the
most challenging component. We then wrote a more general testbench for
the CPU, and after debugging and figuring out some timing issues, we
tested it on nalle running a program calculating fibonacci numbers.

\subsection*{Implementation}

Our implementation consists of the following components:
	
\begin{enumerate}
\item Multiplexer to select register input
\item Program counter
\item Status register
\item Control unit
\item CPU
\item Components given to us from the support files.
\end{enumerate}
	
The program counter is initiated to 0 on reset and is updated every control cycle
with either an increment or an immediate value from the instruction. 
We discussed using a relative jump instead of just loading a value, 
but decided to drop this as there is no need for relative jumps when running 
single programs and we have full control of memory.
  
The status register is just a register with a write enable, and the register 
multiplexer is just a mux placed in a component to clean up our CPU code.  

The control unit sets the control signals to the register mux, the program counter 
and the status register. It is implemented as a Mealey state machine [1]
with a fetch and an execute state. State control is simply switching between theese two
each rising edge of the clock. All the output signals are combinatorialy set based on current
state, input from the current instruction and the status register. For the fetch state all
write signals are set to 0, as the instruction register in the memory is all that is updated
while fetching. For the execute state we set output with a case statement based on the current 
instruction and the status register. See table 1 for control outputs.

\begin{table}[h]
  \centering
  \begin{tabular}{|c|c|c|c|c|} \hline
    Outputs&\emph{ALU}&\emph{LDI}&\emph{BNZ, SR = 1}&\emph{BNZ, SR = 0}\\ \hline
    pc\_mux\_select&0&0&0&1  \\
    reg\_mux\_select&0&1&-&-  \\
    reg\_write\_enable&1&1&0&0  \\
    sr\_write\_enable&1&0&0&0  \\
    \hline
  \end{tabular}
  \caption{Table 1: Outputs in execute state on given input.
  pc\_write\_enable is always 1 in execute state. Dashes represent don't care values}
\end{table}

The module named CPU simply maps signals between all components.  
We chose to implement the CPU exactly as the figure in the assignment text proposed.

We used the instruction set for the CPU as described by table 2, and for the ALU we
used the instruction set implemented in the delivered module.

\begin{table}[h]
  \centering
  \begin{tabular}{|c|c|c|}
    \hline
    Name&Opcode&Comment \\
    \hline
    ALU\_INST&000&Writes alu result to register\\
    BNZ&001&Loads pc register if zero flag is set\\
    LDI&010&Loads register with value\\
    \hline
  \end{tabular}
  \caption{Instruction set for the cpu, not including the alu functionality}
\end{table}

\subsection*{Testing}

%During the implementation process we wrote several testbenches
%for the various components. Of the delivered components we only wrote a testbench
%for the ALU. The testbenches are written to test
%that expected values are outputted on different inputs. If something is wrong, 
%assertions generates warnings when running the test. 
%When all components were ready and connected, we tested the entire 
%design with a testbench for the cpu component (cpu\_testbench.vhd).  
%We wrote this to be the main testbench and implemented it using procedures 
%to fill the memory and check the register values. The cpu testbench loads the 
%fibonacci program, runs and checks for correct register values.


For testing we wrote the following programs:

\subsubsection*{Instruction flush test}

This program tests if the instruction after a branch is successfully flushed.
The program segment loads two values, 246 and 1. It then adds 1 to 246 and
branches back to the add instruction until it reaches 255 and then overflows
to 0. After the overflow add the branch should not be taken and the next instruction
should be executed. If it fails, the register holding 1 will be loaded with 10
and the program will net perform as expected.

\begin{table}[h]
  \centering
  \begin{tabular}{|c|c|c|c|c|c|c|}
    \hline
    Line Nr &	Opcode		&	funct	&	Imm	&	Rd	&	Ra	&	Rb	\\\hline
    	0	&	LDI			&			&	246	&	1	&		&		\\\hline
    	1	&	LDI			&			&	1	&	2	&		&		\\\hline
    	2	&	ALU\_INST	&	ADD		&		&	1	&	2	&	1	\\\hline
    	3	&	BNZ			&			&	2	&		&		&		\\\hline
    	4	&	LDI			&			&	10	&	2	&		&		\\\hline
  \end{tabular}
  \caption{Program to test branching}
\end{table}

\begin{figure}
\centering
\includegraphics[width=.95\linewidth]{test1.png} \\
\caption{This figure shows the waveform from the testbench, 
most of the register signals has been removed, this is because they are not beeing used.}
\end{figure}

\subsubsection*{Datadependency test}

This program test if the dataforwarding is successfull.

\begin{table}[h]
  \centering
  \begin{tabular}{|c|c|c|c|c|c|c|}
    \hline
    Line Nr &	Opcode		&	funct	&	Imm	&	Rd	&	Ra	&	Rb	\\\hline
    	0	&	LDI			&			&	18	&	1	&		&		\\\hline
    	1	&	LDI			&			&	2	&	2	&		&		\\\hline
	\multicolumn{7}{|c|}{Test 1: Dataforwarding from writeback- to execute-stage (one-stage forwarding)}\\\hline
    	2	&	ALU\_INST	&	ADD		&		&	1	&	2	&	1	\\\hline
    	3	&	ALU\_INST	&	ADD		&		&	1	&	2	&	1	\\\hline
    	4	&	ALU\_INST	&	ADD		&		&	1	&	2	&	1	\\\hline
	\multicolumn{7}{|c|}{Test 2: Dataforwarding from writeback- to decode-stage (two-stage forwarding)}\\\hline
    	5	&	ALU\_INST	&	ADD		&		&	1	&	2	&	1	\\\hline
    	6	&	ALU\_INST	&	ADD		&		&	1	&	2	&	2	\\\hline
    	7	&	ALU\_INST	&	ADD		&		&	1	&	2	&	1	\\\hline
  \end{tabular}
  \caption{Program to test data forwarding}
\end{table}

\subsubsection*{Status register test}

This program test if the statusregister is successfully forwarded.
It first loads 254 to register 1, 2 to register 2 and 1 to register 3.
It then adds 254 and 1 and writes it to register 4, the result should because
255. It then adds 2 to 254, this operations should overflow and result in zero,
resulting in the following branch instruction not to be taken. The instructions
following this test should count upwards in the power of 2 until it overflows
and stops. This test is located in file 'cpu\_testbench3.vhd'.

\begin{table}[h]
  \centering
  \begin{tabular}{|c|c|c|c|c|c|c|}
    \hline
    Line Nr &	Opcode		&	funct	&	Imm	&	Rd	&	Ra	&	Rb	\\\hline
    	0	&	LDI			&			&	254	&	1	&		&		\\\hline
    	1	&	LDI			&			&	2	&	2	&		&		\\\hline
    	2	&	LDI			&			&	1	&	3	&		&		\\\hline
    	3	&	ALU\_INST	&	ADD		&		&	4	&	3	&	1	\\\hline
    	4	&	ALU\_INST	&	ADD		&		&	1	&	2	&	1	\\\hline
    	5	&	BNZ			&			&	0	&		&		&		\\\hline
    	6	&	ALU\_INST	&	ADD		&		&	3	&	3	&	3	\\\hline
    	7	&	BNZ			&			&	6	&		&		&		\\\hline

		\end{tabular}
  \caption{Program to test data forwarding}
\end{table}

\begin{figure}
\centering
\includegraphics[width=.95\linewidth]{test3.png} \\
\caption{This figure shows the waveform from the testbench, 
most of the register signals has been removed, this is because they are not beeing used.}
\end{figure}


\subsection*{Synthesis}

From the synthesis report we see that
\begin{enumerate}
\item The regfile muxer inferred 8 1-bit multiplexers, this was a
  little strange as we expected 1 8-bit multiplexer.  This does not
  matter to the functionality though.
\item The sr module inferred one D-type flip flop, this is exatly what
  we wanted.
\item The pc module inferred one 8-bit counter, this is also exactly
  what we wanted.
\item The control module inferred 1 D-type flip flop, this is used to
  hold the current state (fetch or execute), the rest is
  combinatorial, so this is also what we wanted.
\item the toplevel and cpu modules did not infer anything, which was
  expected as they are simply mapping signals between the other
  modules.
\end{enumerate}

The estimated maximum clock frequency was 61.747MHz.

Selected Device : v1000efg860-6 

Usage of selected device:
\begin{table}[h]
  \centering
  \begin{tabular}{|l|l|l|} 
    \hline
    Number of Slices:&252 out of 12288&2\% \\ 
    Number of Slice Flip Flops:&205 out of 24576&0\% \\
    Number of 4 input LUTs:&454 out of 24576&1\% \\
    Number of bonded IOBs:&54 out of 664&8\% \\ 
    Number of BRAMs:&2 out of 16&12\% \\  
    Number of GCLKs:&2 out of 4&50\% \\
    \hline
  \end{tabular}
\end{table}


We got no errors from the place and route report, all constraints met,
and all signals routed.

\section*{Result}

% Present the result of the work. What works and what does not. 
% What the problems were if there were
% any. Or even better: what was fantastic in the group's work.
% Here is the place to add some screen-shots, debug outputs or similar to
% support the documentation. It is also the place to praise or show more
% critique towards the presented work.

Everything seemed to work after a couple of days work, 
both simulation and testing on nalle.

However, we had previously tried some different solutions
which did not work, and we realized that we did not understand why they did 
not work. We first tried to write to the registers on the rising edge of the clock, 
but this did not work. We then used the write enable signal as the clock 
on the registers and everything worked great.
However, we changed this later, when we discovered what we did wrong earlier,
which was having registers on the output of the control unit. After that
we removed the registers on the control unit and set the registers to be
written on the rising edge of the clock, and everything worked the way we
intended it to.

\section*{Evaluation}

% If there is something about the exercise what should be changed, 
% mention it here – text in the
% compendium, support files, exercise itself or something else which
% us as teaching staff should see to.

This assignment was very well defined compared to the previous. There was
alot less guesswork involved during the design process. We found this to
be more educational as we could focus on the tasks, in contrast to
making tasks up.

\section*{Conclusion}

% Round up - what works, what does not work. Comment on the exercise. 
% Write a bit about what was easy and what not so.

It was easy to map everything together and create most of the components. 
The difficult part of this assignment was to figure out the timing 
and syncronization, what signals to register and when they where 
ready to be read. All in all this was a fun and educational exercise.

\section*{References}

[1] http://en.wikipedia.org/wiki/Mealy\_machine

\end{document}


